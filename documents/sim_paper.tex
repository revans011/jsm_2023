% Options for packages loaded elsewhere
\PassOptionsToPackage{unicode}{hyperref}
\PassOptionsToPackage{hyphens}{url}
%
\documentclass[
]{article}
\usepackage{amsmath,amssymb}
\usepackage{iftex}
\ifPDFTeX
  \usepackage[T1]{fontenc}
  \usepackage[utf8]{inputenc}
  \usepackage{textcomp} % provide euro and other symbols
\else % if luatex or xetex
  \usepackage{unicode-math} % this also loads fontspec
  \defaultfontfeatures{Scale=MatchLowercase}
  \defaultfontfeatures[\rmfamily]{Ligatures=TeX,Scale=1}
\fi
\usepackage{lmodern}
\ifPDFTeX\else
  % xetex/luatex font selection
\fi
% Use upquote if available, for straight quotes in verbatim environments
\IfFileExists{upquote.sty}{\usepackage{upquote}}{}
\IfFileExists{microtype.sty}{% use microtype if available
  \usepackage[]{microtype}
  \UseMicrotypeSet[protrusion]{basicmath} % disable protrusion for tt fonts
}{}
\makeatletter
\@ifundefined{KOMAClassName}{% if non-KOMA class
  \IfFileExists{parskip.sty}{%
    \usepackage{parskip}
  }{% else
    \setlength{\parindent}{0pt}
    \setlength{\parskip}{6pt plus 2pt minus 1pt}}
}{% if KOMA class
  \KOMAoptions{parskip=half}}
\makeatother
\usepackage{xcolor}
\usepackage[margin=1.0in]{geometry}
\usepackage{longtable,booktabs,array}
\usepackage{calc} % for calculating minipage widths
% Correct order of tables after \paragraph or \subparagraph
\usepackage{etoolbox}
\makeatletter
\patchcmd\longtable{\par}{\if@noskipsec\mbox{}\fi\par}{}{}
\makeatother
% Allow footnotes in longtable head/foot
\IfFileExists{footnotehyper.sty}{\usepackage{footnotehyper}}{\usepackage{footnote}}
\makesavenoteenv{longtable}
\usepackage{graphicx}
\makeatletter
\def\maxwidth{\ifdim\Gin@nat@width>\linewidth\linewidth\else\Gin@nat@width\fi}
\def\maxheight{\ifdim\Gin@nat@height>\textheight\textheight\else\Gin@nat@height\fi}
\makeatother
% Scale images if necessary, so that they will not overflow the page
% margins by default, and it is still possible to overwrite the defaults
% using explicit options in \includegraphics[width, height, ...]{}
\setkeys{Gin}{width=\maxwidth,height=\maxheight,keepaspectratio}
% Set default figure placement to htbp
\makeatletter
\def\fps@figure{htbp}
\makeatother
\setlength{\emergencystretch}{3em} % prevent overfull lines
\providecommand{\tightlist}{%
  \setlength{\itemsep}{0pt}\setlength{\parskip}{0pt}}
\setcounter{secnumdepth}{-\maxdimen} % remove section numbering
\usepackage{helvet}
\renewcommand*\familydefault{\sfdefault}
\usepackage{setspace}
\doublespacing
\usepackage[left]{lineno}
\linenumbers
\usepackage{booktabs}
\usepackage{longtable}
\usepackage{array}
\usepackage{multirow}
\usepackage{wrapfig}
\usepackage{float}
\usepackage{colortbl}
\usepackage{pdflscape}
\usepackage{tabu}
\usepackage{threeparttable}
\usepackage{threeparttablex}
\usepackage[normalem]{ulem}
\usepackage{makecell}
\usepackage{xcolor}
\ifLuaTeX
  \usepackage{selnolig}  % disable illegal ligatures
\fi
\IfFileExists{bookmark.sty}{\usepackage{bookmark}}{\usepackage{hyperref}}
\IfFileExists{xurl.sty}{\usepackage{xurl}}{} % add URL line breaks if available
\urlstyle{same}
\hypersetup{
  hidelinks,
  pdfcreator={LaTeX via pandoc}}

\author{}
\date{\vspace{-2.5em}}

\begin{document}

\hypertarget{sample-size-considerations-in-the-design-of-orthopaedic-risk-factor-studies}{%
\section{Sample Size Considerations in the Design of Orthopaedic
Risk-factor
Studies}\label{sample-size-considerations-in-the-design-of-orthopaedic-risk-factor-studies}}

\textbf{Running title:} Sample Size Considerations

Richard Evans \({^\dagger}\)

\({\dagger}\) Corresponding Author

Clinical and Translational Science Institute\\
University of Minnesota

\href{mailto:evan0770@umn.edu}{evan0770@umn.edu}

\newpage

\#\#Abstract

\hypertarget{objective}{%
\subsubsection{Objective}\label{objective}}

Sample size calculations play a pivotal role in study design, as they
influence study interpretability, costs, and the allocation of hospital
resources and staff time. In the context of most orthopedic risk-factor
studies, either the sample size calculation or the post-hoc power
calculation assumes the precise ascertainment of disease status among
control subjects, which might not always hold true. Consequently,
control groups might consist of a mixture of both unaffected cases and
some unidentified affected cases. Negative control groups containing
misclassified positive data are denoted as ``unlabeled.'' Treating
unlabeled groups as disease-negative control groups is recognized to
introduce misclassification bias. However, scant research has been
conducted on the impact of such misclassification on the statistical
power of risk association tests. In this study, we elucidate the
repercussions of employing unlabeled groups as control groups on the
power of risk association tests. Our aim is to demonstrate that even
minor misclassification rates within control groups can substantially
diminish the power of association tests. Consequently, disregarding the
unlabeled aspect of control groups in sample size calculations may lead
to underpowered studies. Additionally, we present a range of correction
factors to recalibrate sample size calculations to achieve 80\% power.

\hypertarget{materials-and-methods}{%
\subsubsection{Materials and Methods}\label{materials-and-methods}}

This study employed a simulation approach, utilizing study designs from
published orthopedic risk-factor studies. The methodology involved
adopting these designs and subsequently simulating the data to
incorporate predetermined proportions of misclassified affected subjects
within the control group. The simulated dataset was then employed to
compute the power of a risk-association test. We calculated the
statistical power for various study designs and misclassification rates,
subsequently comparing these results against a reference model.

\hypertarget{results}{%
\subsubsection{Results}\label{results}}

Treating unlabeled data as disease-negative consistently resulted in a
reduction of statistical power in comparison to the reference power.
Moreover, the extent of power loss increased as the misclassification
rate escalated. In the context of this study, restoring the statistical
power to 80\% was attainable by increasing the sample size by a factor
ranging from 1.1 to 1.4.

\hypertarget{conclusion}{%
\subsubsection{Conclusion}\label{conclusion}}

Researchers should exercise caution when calculating sample sizes for
risk-factor studies and should incorporate adjustments for estimated
misclassification rates.

Keywords: risk-factor, case-control, power, sample size

\newpage

\hypertarget{introduction}{%
\subsection{Introduction}\label{introduction}}

Sample size calculations are pivotal in study design, influencing study
costs, allocation of hospital resources, and staff time. Underpowered
studies, fraught with Type II errors, can be challenging to interpret
and pose ethical concerns with minimal prospects of success.
\cite{ halpern2002continuing, hofmeister2007sample}.

In orthopedic risk factor studies, while the positive disease status of
affected subjects is ascertained with perfect sensitivity and
specificity, control subjects' disease status is sometimes uncertain.
This results in control groups comprising a blend of unaffected and
unidentified affected cases. Control groups with misclassified data are
termed ``unlabeled.'' Such data scenarios, featuring truly affected
cases in the positive group and an unlabeled control group, are referred
to as ``positive-unlabeled'' (PU) data within the data science
community.

Although examples of positive-unlabeled data are well-documented in
human medicine, their presence in veterinary medicine is less explored.
Nevertheless, many veterinary studies fall within the positive-unlabeled
framework. For instance, genome-wide association studies on cranial
cruciate ligament disease (CCLD) in dogs adopt case-control designs. The
affected cases, derived from dogs undergoing knee stabilization surgery,
are unequivocally positive for CCLD. Control cases are typically five
years old or older, displaying no CCLD history, and clearing an
orthopedic veterinary examination by a board-certified surgeon. However,
some control dogs might experience future spontaneous ruptures, thus
genetically aligning with the CCLD-affected group. Other controls might
exhibit sub-diagnostic disease, masking latent CCLD. \cite{wrehim08}

There are other examples of PU data in the veterinary literature,
typically in risk-factor studies using case-control designs. For
example, Arthur et al.~(2016) used a case-control design to assess the
risk of osteosarcoma following fracture repair. \cite{aakj16} They said,
``There may be additional cases {[}in the control group{]} in which
implant-related osteosarcoma was diagnosed in the private practice
setting without referral\ldots,'' suggesting that the control group may
be unlabeled because some control-group cases were actually osteosarcoma
positive, but diagnosed outside the study. In another example, Wylie et
al.~2013 studied risk factors for equine laminitis using controls
obtained from an owner survey. \cite{wcvj13} The authors noted the
positive-unlabeled aspect of their data, ``Our study relied on
owner-reported diagnoses of endocrinopathic conditions, and this may
have introduced misclassification bias.''

Abstract Objective Sample size calculations are pivotal in study design,
influencing study costs, allocation of hospital resources, and staff
time. Underpowered studies, fraught with Type II errors, can be
challenging to interpret and pose ethical concerns with minimal
prospects of success (Halpern, 2002; Hofmeister, 2007).

In orthopedic risk factor studies, while the positive disease status of
affected subjects is ascertained with perfect sensitivity and
specificity, control subjects' disease status is sometimes uncertain.
This results in control groups comprising a blend of unaffected and
unidentified affected cases. Control groups with misclassified data are
termed ``unlabeled.'' Such data scenarios, featuring truly affected
cases in the positive group and an unlabeled control group, are referred
to as ``positive-unlabeled'' (PU) data within the data science
community.

Although examples of positive-unlabeled data are well-documented in
human medicine, their presence in veterinary medicine is less explored.
Nevertheless, many veterinary studies fall within the positive-unlabeled
framework. For instance, genome-wide association studies on cranial
cruciate ligament disease (CCLD) in dogs adopt case-control designs. The
affected cases, derived from dogs undergoing knee stabilization surgery,
are unequivocally positive for CCLD. Control cases are typically five
years old or older, displaying no CCLD history, and clearing an
orthopedic veterinary examination by a board-certified surgeon. However,
some control dogs might experience future spontaneous ruptures, thus
genetically aligning with the CCLD-affected group. Other controls might
exhibit sub-diagnostic disease, masking latent CCLD.

As mentioned above, the affected cases are ``labeled'' positive, but the
control data is ``unlabeled,'' because dogs may be affected or
unaffected. Treating the unlabeled control group as entirely unaffected
is called the \emph{naive model}. The proportion of affected dogs in the
control group is called the \emph{nondetection rate} or \emph{undetected
rate}.

Using the naive model when the nondetection rate is positive causes
misclassification bias (because there affected cases in the control
group), and that bias is well documented in the data science literature.
\cite{bd20} Biases due to misclassification can be mitigated using
models other than the naive model and with the appropriate data
analysis, and there are many articles describing methods for analyzing
positive-unlabeled data. Bekker and Davis (2020) provides and excellent
summary of methods. \cite{bd20} Sometimes, however, researchers prefer
the naive model because the analysis is simpler and they believe their
small nondetection rates induce misclassification biases that are too
small for practical consideration. There is some suggestion that
nondetection rates under 10\% do have little impact on bias. \cite{bd20}

But bias in estimates (e.g., bias in regression coefficients) is just
one part of the results; the other part is inference (e.g., p-values).
Central to inference is the power of statistical tests. Power is used in
planning a study as a measure of the ability of the study to make the
correct decisions. That is, finding P\textless0.05 when it should.
Typically, 80 percent power means that if the group parameters are truly
different, then the statistical test has an 80 percent chance of
obtaining p\textless0.05.

During the design phase of risk association studies, researchers often
calculate the sample size required for 80\% power assuming a zero
nondetection rate. In other words, they presume no misclassified
affected subjects in the control group. However, if collected data
conform to the positive-unlabeled scenario, the naive model becomes
inappropriate, and estimated power might be lower than anticipated.

This study examines the impact of positive-unlabeled data on the loss of
statistical power under the naive model. For comparison, the reference
power is defined when the naive model is accurate, and group sizes are
balanced. Results quantify power loss relative to reference power in
terms of both percentage and absolute power loss---parallel to relative
and absolute risk in epidemiology.

Through simulations, we elucidate how statistical power varies based on
different proportions of undetected positives in naive controls and
varying imbalances between case and control numbers. Our first objective
is to demonstrate that naive analysis of positive-unlabeled data
diminishes statistical power in risk-factor studies, even with minor
nondetection rates. Our second objective is to provide correction
factors to adjust sample sizes upwards, rectifying the power loss
outlined in the first objective.

\newpage

\hypertarget{methods-and-materials}{%
\subsection{Methods and materials}\label{methods-and-materials}}

\hypertarget{the-test-of-association}{%
\subsubsection{The Test of Association}\label{the-test-of-association}}

This study encompassed a simulation-based approach aimed at evaluating
the alterations in the power of a univariate association test across
various positive-unlabeled (PU) conditions. While numerous statistical
tests gauge association, our analysis specifically focused on one of the
most prevalent tests, namely the Fisher exact test. This test is
commonly employed to assess the statistical significance of a binary
risk factor. Broadly, the test can be extended to evaluate the
significance of any risk factor by utilizing predicted values derived
from a univariate logistic regression.

Within the realm of risk association studies, assuming all other factors
remain constant, the Fisher exact test achieves its peak power in a
balanced study design when the naive model holds true (i.e., no
undetected positives in the control group). This maximum power is termed
the ``reference power,'' and we present our findings in terms of both
the percentage of power loss relative to the reference power and the
absolute power loss from the reference power. In essence, we employed
the Fisher exact test to illustrate the extent of statistical power
reduction resulting from the oversight of the nondetection rate.

\hypertarget{the-sample-size-and-group-imbalance}{%
\subsubsection{The Sample Size and Group
Imbalance}\label{the-sample-size-and-group-imbalance}}

The total sample size for the simulation was fixed N=200, which is
consistent with Healey et al.~2019 (N=216), and Baird et al.~2014
(N=217). \cite{bcioa14} \cite{hmhcbhkr19} The effect size, 0.21, was
chosen because with N=200, the reference power was close to 80 percent,
which is value that is commonly used in study design. That way, the
reference model is the one with standard power of 80 percent. Note that
the sample size and effect size are not a key parameters for the
simulation because for any sample size an effect size can be chosen so
that power is 80 percent. Also, effect size and sample size are not
features of PU data, per se.

The simulation study varied two study design parameters: the
nondetection rate and group-size imbalance. The proportion of undetected
positives in the control group ranged from 0 (the value for reference
power) to 10 percent. We used 10 percent as the upper limit because
researchers are generally willing to accept nondetection rates below 10
percent and use the naive model, but change to a PU analysis for rates
greater than 10 percent. \cite{bd20}

We modeled group imbalance using Healey et al.~(2019), which used 161
dogs affected with CCLD and 55 unlabeled dogs as controls, and Baird et
al.~(2014) which used 91 dogs affected with CCLD, and 126 unlabeled dogs
as controls, so that imbalance ratios were about 3:1 and 1:3.
\cite{bcioa14} \cite{hmhcbhkr19} We only used two imbalance proportions
(1:3 and 3:1) and no imbalance (1:1) because the key parameter for this
study was the nondetection proportion. That gave simulation sample sizes
of (50, 150), (150, 50), and (100, 100).

\pagebreak

\hypertarget{the-simulation-algorithm}{%
\subsubsection{The Simulation
Algorithm}\label{the-simulation-algorithm}}

The overall approach is to simulate data, and then use that data to
calculate the p-value of the Fisher exact test. The process is repeated
5000 times for each combination of sample size and nondetection rate,
and then the 5000 p-values are compared to 0.05. The proportion of
p-values less than 0.05 is the estimated power.

Simulating the data works backward from what might be expected. Instead
starting with values for a risk factor (e.g., 200 0's and 1's
representing sex) and then simulating their disease status, we start
with the disease status (e.g., 50 affected cases and 150 controls with
135 unaffected and 15 affected) and then assign binary values for the
risk factor. It was done that way to control the nondetection rate and
group sizes.

The simulation algorithm is most easily described using examples, and we
begin with calculating power for the Fisher exact test under the
reference model, which is 100 cases and 100 correctly labeled (i.e., 100
truly unaffected) controls. That is, there are no affected cases in the
control group, so this is not positive-unlabeled data, and the naive
model is the correct model. Next we associate a binary risk factor
variable, \(X\) (e.g., sex), with the cases and controls.The negative
controls were simulated by sampling 100 negative cases from a binomial
distribution with \(Pr(X=1) = 0.2\). That probability means the the
baseline risk for the disease in the population is 0.2. It it was chosen
arbitrarily, because the baseline risk isn't central to power, the
effect size is. As mentioned above, the effect size was 0.21, so the 100
positive cases were sampled from a binomial distribution with
\(Pr(X=1) = 0.2 + 0.21\). Using the sex example, that means that having
sex=1 predisposes the animals to about double the baseline risk of
disease (the baseline is 0.2, and with sex = 1, the risk is double, 0.2+
+ 0.21 = 0.41.).

Now, the 200 cases are pairs of binary data, one representing the group
and the other representing the risk factor. These simulated data were
tested with the Fisher exact test. As mentioned above, this process was
repeated 5000 times and the resulting 5000 p-values used to estimate
power.

For the second example, we calculate the power for a positive-unlabeled
example. Suppose that in a 100-patient control group, 10 percent are in
fact undetected positives. So the dataset is 10 affected cases in the
control group, 90 unaffected cases in the control group, and 100
affected cases in the positive group. As in the previous example, the
risk factor is simulated by sampling from binomial distributions. Now,
90 controls are sampled from the binomial distribution with
\(Pr(X=1) = 0.2\), the 10 affected controls are sampled from the
binomial distribution with \(Pr(X=1) = 0.2+ 0.21\), and 100 cases are
sampled from the same binomial distribution with
\(Pr(X=1) = 0.2 + 0.21\). The 10 mislabeled affected cases remain in the
control group, so as to measure the effect of treating PU data naively.
As before, the simulated data are treated like pilot data and p-values
were calculated. This process is repeated 5000 times and the was
estimated as described in the previous example.

\hypertarget{the-correction-factor}{%
\subsubsection{The Correction Factor}\label{the-correction-factor}}

For aim 2, the sample-size correction factor estimation, we used the
same simulation algorithm and effect size (0.21) but multiplied the
group sample sizes by possible correction factors, 1.1, 1.2, and so on,
increasing sample size and therefore the power, until it reached the
80\%.

\pagebreak

\hypertarget{results-1}{%
\subsection{Results}\label{results-1}}

Table 1 describes power loss for the three study designs with three
different group sizes, (50, 150), (150, 50), and (100, 100), and for
three nondetection rates, 0, 0.05 (5\%), and 0.1 (10\%). To give these
parameters context, if the group sizes are (50, 150) and the
nondectection rate is 0.1, then the positive (affected) group has 50
cases, and the unlabeled control group (which we are treating naively in
the analysis) has 150 cases, 15 of which are actually affected cases.
When the nondetection rate is zero, the naive model is correct because
there no affected cases in the control group. The first row is the
reference power, so its loss of power compared to itself is zero. The
reference power was calculated in the simulation just like all the other
powers, and was estimated to be 0.82.

Columns 5 and 6 are the power loss columns and have negative entries
because for this simulation positive-unlabeled data analyzed under the
naive model always had lower power, as did unbalanced data. Column 5 is
the percent loss from the reference power (82\%) and column 6 is the
absolute power loss from the reference power. For example, the second
row shows a -4.81\% relative power reduction when the group sizes are
balanced but five percent (0.05) of the control group are actually
positive cases.

Rows one, four and seven are correct models (i.e., no positives in the
control group). Rows four and seven show a power loss due to sample size
imbalance only. So, for this small example, group imbalance sometimes
caused more power loss than misclassified data as is seen by comparing
row 3 to row four. It is known that for equal overall sample size, group
imbalance results in less powerful tests. As an aside, more data is
often better than less data, and it is sometimes better to have more
unbalanced data than fewer balanced data.

Using Table 1, increasing nondetection rate within a study design
decreased power. For example, for the (100, 100) study design, power
decreased by more than 10\% as the non-detection rate increased (row one
to three). For the (50, 150) design, rows seven to nine, power decreased
by 12.02\% (22.47 - 10.45) from the correct model (row seven), but
22.47\% from the reference model.

Finally, note that for this simulation, the absolute power loses are
marked, but not extreme. For example, in the (100, 100) design, (rows 1
to 3) the power dropped to 0.73 (0.82 - 0.09) for row 3.

\pagebreak

Table 1. Power loss. This table orders sample sizes by relative power
loss (\%). The first row is the reference power, which had an absolute
power of 0.82 (82\%). The last two columns represent power loss relative
to 0.82, both as a percentage and absolute difference. Note that some
inconsistancies in the table are due to rounding. For example, in rows 3
to 5, the absolute power is constant at 0.09, but the relative power
changes.

\begin{longtable}[]{@{}
  >{\raggedleft\arraybackslash}p{(\columnwidth - 10\tabcolsep) * \real{0.0339}}
  >{\raggedleft\arraybackslash}p{(\columnwidth - 10\tabcolsep) * \real{0.1441}}
  >{\raggedleft\arraybackslash}p{(\columnwidth - 10\tabcolsep) * \real{0.1441}}
  >{\raggedleft\arraybackslash}p{(\columnwidth - 10\tabcolsep) * \real{0.2034}}
  >{\raggedleft\arraybackslash}p{(\columnwidth - 10\tabcolsep) * \real{0.2034}}
  >{\raggedleft\arraybackslash}p{(\columnwidth - 10\tabcolsep) * \real{0.2712}}@{}}
\toprule\noalign{}
\begin{minipage}[b]{\linewidth}\raggedleft
Row
\end{minipage} & \begin{minipage}[b]{\linewidth}\raggedleft
N positive cases
\end{minipage} & \begin{minipage}[b]{\linewidth}\raggedleft
N naive controls
\end{minipage} & \begin{minipage}[b]{\linewidth}\raggedleft
nondetection proportion
\end{minipage} & \begin{minipage}[b]{\linewidth}\raggedleft
Relative power loss (\%)
\end{minipage} & \begin{minipage}[b]{\linewidth}\raggedleft
Absolute power loss (from 0.82)
\end{minipage} \\
\midrule\noalign{}
\endhead
\bottomrule\noalign{}
\endlastfoot
1 & 100 & 100 & 0.00 & 0.00 & 0.00 \\
2 & 100 & 100 & 0.05 & -4.81 & -0.04 \\
3 & 100 & 100 & 0.10 & -10.29 & -0.09 \\
4 & 150 & 50 & 0.00 & -10.77 & -0.09 \\
5 & 150 & 50 & 0.05 & -14.92 & -0.13 \\
6 & 150 & 50 & 0.10 & -22.50 & -0.20 \\
7 & 50 & 150 & 0.00 & -10.45 & -0.09 \\
8 & 50 & 150 & 0.05 & -15.26 & -0.13 \\
9 & 50 & 150 & 0.10 & -22.47 & -0.20 \\
\end{longtable}

\pagebreak

Table 2 hows how many additional subject are require to regain power
when the non-detection rate is 10\%. Rows 1 to 5 are for the (100, 100)
design, rows 6 to 10 are for the (150, 50) design, and rows 11 to 15 are
for the (50, 150) design. Sample size was increased by 10\% for each row
within a study design. As one might expect, lower positive-unlabeled
power needs more subjects to bring the power up to 80\%. For the
unbalanced designs, the increased sample size also fixes the power loss
due to imbalance. For example, in row one, the power is 0.69 for the
original (50, 150) design with 10\% nondetection rate. Row 14 shows that
an additional 65 subjects, or 32.5\% more subjects are required to bring
the power above 80\%. However, for the (100, 100) design, only 10\% more
subjects are required (rows one and two).

Table 2. Power improvement. Rows 1, 6, and 11, show the power for when
there are no false positives. The other rows show improvements in power
when there is a 10\% nondection rate. The sixth column shows the percent
increase in sample size, and the last column is power.

\begin{longtable}[]{@{}
  >{\raggedleft\arraybackslash}p{(\columnwidth - 12\tabcolsep) * \real{0.0440}}
  >{\raggedleft\arraybackslash}p{(\columnwidth - 12\tabcolsep) * \real{0.1868}}
  >{\raggedleft\arraybackslash}p{(\columnwidth - 12\tabcolsep) * \real{0.1868}}
  >{\raggedleft\arraybackslash}p{(\columnwidth - 12\tabcolsep) * \real{0.1868}}
  >{\raggedleft\arraybackslash}p{(\columnwidth - 12\tabcolsep) * \real{0.0879}}
  >{\raggedleft\arraybackslash}p{(\columnwidth - 12\tabcolsep) * \real{0.2418}}
  >{\raggedleft\arraybackslash}p{(\columnwidth - 12\tabcolsep) * \real{0.0659}}@{}}
\toprule\noalign{}
\begin{minipage}[b]{\linewidth}\raggedleft
Row
\end{minipage} & \begin{minipage}[b]{\linewidth}\raggedleft
N positive cases
\end{minipage} & \begin{minipage}[b]{\linewidth}\raggedleft
N naive controls
\end{minipage} & \begin{minipage}[b]{\linewidth}\raggedleft
N false controls
\end{minipage} & \begin{minipage}[b]{\linewidth}\raggedleft
N total
\end{minipage} & \begin{minipage}[b]{\linewidth}\raggedleft
percent increase in N
\end{minipage} & \begin{minipage}[b]{\linewidth}\raggedleft
Power
\end{minipage} \\
\midrule\noalign{}
\endhead
\bottomrule\noalign{}
\endlastfoot
1 & 100 & 100 & 10 & 200 & 0.0 & 0.78 \\
2 & 110 & 110 & 11 & 220 & 10.0 & 0.83 \\
3 & 120 & 120 & 12 & 240 & 20.0 & 0.87 \\
4 & 130 & 130 & 13 & 260 & 30.0 & 0.89 \\
5 & 140 & 140 & 14 & 280 & 40.0 & 0.90 \\
6 & 150 & 50 & 5 & 200 & 0.0 & 0.67 \\
7 & 165 & 55 & 6 & 220 & 10.0 & 0.70 \\
8 & 180 & 60 & 6 & 240 & 20.0 & 0.77 \\
9 & 195 & 79 & 8 & 274 & 37.0 & 0.84 \\
10 & 210 & 80 & 8 & 290 & 45.0 & 0.87 \\
11 & 50 & 150 & 15 & 200 & 0.0 & 0.69 \\
12 & 55 & 165 & 16 & 220 & 10.0 & 0.73 \\
13 & 60 & 180 & 18 & 240 & 20.0 & 0.78 \\
14 & 70 & 195 & 20 & 265 & 32.5 & 0.83 \\
15 & 80 & 210 & 21 & 290 & 45.0 & 0.85 \\
\end{longtable}

\pagebreak

\hypertarget{discussion}{%
\section{Discussion}\label{discussion}}

This study demonstrated that under specific conditions, there was a
modest decline in power, even with relatively small proportions of
undetected positives in the control group. This implies that risk-factor
studies might exhibit lower-than-expected power, consequently elevating
the risk of encountering Type II errors. Importantly, it should be noted
that changes in power can influence more than just statistical power
itself. For instance, consider Table 1, rows 1 to 3, where the absolute
power decreased from 0.82 (row 1) to 0.73 (row 3). To regain the
reference power (as per Table 2), approximately 20 additional subjects
would be required. In cases where subjects are expensive, the seemingly
minor power drop of 0.09 translates into significant cost implications.
Conversely, for an exploratory retrospective study, a power drop of 0.09
(equivalent to 9\%) might not be considered substantial.

The working examples were derived from Genome-Wide Association Studies
(GWAS), yet the simulation outcomes are applicable to any study
utilizing univariate association tests, including various risk factor
studies. This encompasses a wide spectrum of study types. For instance,
it covers the univariate associations between post-operative surgical
infections and different surgical conditions (e.g., board-certified
surgeon vs.~resident, bone plate manufacturer). In such scenarios, the
control group may harbor subdiagnostic infections. Similarly, univariate
association tests in veterinary surveys also fall within this category.

In this simulation, undetected positive cases in the negative group were
randomly sampled from the same population as detected positives in the
affected group. Although this is a common assumption, alternative models
exist (Greenland \& Rothman, 2004). In one such model, undetected
positive cases in the control group could constitute a subpopulation of
positives defined by another variable. For example, in a GWAS study
focusing on cranial cruciate ligament disease (CCLD), the undetected
positive cases in the control group might consist of positive dogs with
low body condition scores. However, such models were not explored in
this research. Our objective was to identify instances illustrating that
in certain studies, misclassified data can lead to power loss. When this
power loss is compounded with imbalanced data, the resulting loss can be
substantial.

\newpage

\bibliographystyle{unsrt}

\bibliography{references}

\end{document}
